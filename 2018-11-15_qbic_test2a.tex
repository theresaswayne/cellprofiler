\documentclass[]{article}
\usepackage{lmodern}
\usepackage{amssymb,amsmath}
\usepackage{ifxetex,ifluatex}
\usepackage{fixltx2e} % provides \textsubscript
\ifnum 0\ifxetex 1\fi\ifluatex 1\fi=0 % if pdftex
  \usepackage[T1]{fontenc}
  \usepackage[utf8]{inputenc}
\else % if luatex or xelatex
  \ifxetex
    \usepackage{mathspec}
  \else
    \usepackage{fontspec}
  \fi
  \defaultfontfeatures{Ligatures=TeX,Scale=MatchLowercase}
\fi
% use upquote if available, for straight quotes in verbatim environments
\IfFileExists{upquote.sty}{\usepackage{upquote}}{}
% use microtype if available
\IfFileExists{microtype.sty}{%
\usepackage{microtype}
\UseMicrotypeSet[protrusion]{basicmath} % disable protrusion for tt fonts
}{}
\usepackage[margin=1in]{geometry}
\usepackage{hyperref}
\hypersetup{unicode=true,
            pdftitle={2018-11-15 qbic test 2a},
            pdfauthor={Theresa Swayne},
            pdfborder={0 0 0},
            breaklinks=true}
\urlstyle{same}  % don't use monospace font for urls
\usepackage{color}
\usepackage{fancyvrb}
\newcommand{\VerbBar}{|}
\newcommand{\VERB}{\Verb[commandchars=\\\{\}]}
\DefineVerbatimEnvironment{Highlighting}{Verbatim}{commandchars=\\\{\}}
% Add ',fontsize=\small' for more characters per line
\usepackage{framed}
\definecolor{shadecolor}{RGB}{248,248,248}
\newenvironment{Shaded}{\begin{snugshade}}{\end{snugshade}}
\newcommand{\KeywordTok}[1]{\textcolor[rgb]{0.13,0.29,0.53}{\textbf{#1}}}
\newcommand{\DataTypeTok}[1]{\textcolor[rgb]{0.13,0.29,0.53}{#1}}
\newcommand{\DecValTok}[1]{\textcolor[rgb]{0.00,0.00,0.81}{#1}}
\newcommand{\BaseNTok}[1]{\textcolor[rgb]{0.00,0.00,0.81}{#1}}
\newcommand{\FloatTok}[1]{\textcolor[rgb]{0.00,0.00,0.81}{#1}}
\newcommand{\ConstantTok}[1]{\textcolor[rgb]{0.00,0.00,0.00}{#1}}
\newcommand{\CharTok}[1]{\textcolor[rgb]{0.31,0.60,0.02}{#1}}
\newcommand{\SpecialCharTok}[1]{\textcolor[rgb]{0.00,0.00,0.00}{#1}}
\newcommand{\StringTok}[1]{\textcolor[rgb]{0.31,0.60,0.02}{#1}}
\newcommand{\VerbatimStringTok}[1]{\textcolor[rgb]{0.31,0.60,0.02}{#1}}
\newcommand{\SpecialStringTok}[1]{\textcolor[rgb]{0.31,0.60,0.02}{#1}}
\newcommand{\ImportTok}[1]{#1}
\newcommand{\CommentTok}[1]{\textcolor[rgb]{0.56,0.35,0.01}{\textit{#1}}}
\newcommand{\DocumentationTok}[1]{\textcolor[rgb]{0.56,0.35,0.01}{\textbf{\textit{#1}}}}
\newcommand{\AnnotationTok}[1]{\textcolor[rgb]{0.56,0.35,0.01}{\textbf{\textit{#1}}}}
\newcommand{\CommentVarTok}[1]{\textcolor[rgb]{0.56,0.35,0.01}{\textbf{\textit{#1}}}}
\newcommand{\OtherTok}[1]{\textcolor[rgb]{0.56,0.35,0.01}{#1}}
\newcommand{\FunctionTok}[1]{\textcolor[rgb]{0.00,0.00,0.00}{#1}}
\newcommand{\VariableTok}[1]{\textcolor[rgb]{0.00,0.00,0.00}{#1}}
\newcommand{\ControlFlowTok}[1]{\textcolor[rgb]{0.13,0.29,0.53}{\textbf{#1}}}
\newcommand{\OperatorTok}[1]{\textcolor[rgb]{0.81,0.36,0.00}{\textbf{#1}}}
\newcommand{\BuiltInTok}[1]{#1}
\newcommand{\ExtensionTok}[1]{#1}
\newcommand{\PreprocessorTok}[1]{\textcolor[rgb]{0.56,0.35,0.01}{\textit{#1}}}
\newcommand{\AttributeTok}[1]{\textcolor[rgb]{0.77,0.63,0.00}{#1}}
\newcommand{\RegionMarkerTok}[1]{#1}
\newcommand{\InformationTok}[1]{\textcolor[rgb]{0.56,0.35,0.01}{\textbf{\textit{#1}}}}
\newcommand{\WarningTok}[1]{\textcolor[rgb]{0.56,0.35,0.01}{\textbf{\textit{#1}}}}
\newcommand{\AlertTok}[1]{\textcolor[rgb]{0.94,0.16,0.16}{#1}}
\newcommand{\ErrorTok}[1]{\textcolor[rgb]{0.64,0.00,0.00}{\textbf{#1}}}
\newcommand{\NormalTok}[1]{#1}
\usepackage{longtable,booktabs}
\usepackage{graphicx,grffile}
\makeatletter
\def\maxwidth{\ifdim\Gin@nat@width>\linewidth\linewidth\else\Gin@nat@width\fi}
\def\maxheight{\ifdim\Gin@nat@height>\textheight\textheight\else\Gin@nat@height\fi}
\makeatother
% Scale images if necessary, so that they will not overflow the page
% margins by default, and it is still possible to overwrite the defaults
% using explicit options in \includegraphics[width, height, ...]{}
\setkeys{Gin}{width=\maxwidth,height=\maxheight,keepaspectratio}
\IfFileExists{parskip.sty}{%
\usepackage{parskip}
}{% else
\setlength{\parindent}{0pt}
\setlength{\parskip}{6pt plus 2pt minus 1pt}
}
\setlength{\emergencystretch}{3em}  % prevent overfull lines
\providecommand{\tightlist}{%
  \setlength{\itemsep}{0pt}\setlength{\parskip}{0pt}}
\setcounter{secnumdepth}{0}
% Redefines (sub)paragraphs to behave more like sections
\ifx\paragraph\undefined\else
\let\oldparagraph\paragraph
\renewcommand{\paragraph}[1]{\oldparagraph{#1}\mbox{}}
\fi
\ifx\subparagraph\undefined\else
\let\oldsubparagraph\subparagraph
\renewcommand{\subparagraph}[1]{\oldsubparagraph{#1}\mbox{}}
\fi

%%% Use protect on footnotes to avoid problems with footnotes in titles
\let\rmarkdownfootnote\footnote%
\def\footnote{\protect\rmarkdownfootnote}

%%% Change title format to be more compact
\usepackage{titling}

% Create subtitle command for use in maketitle
\newcommand{\subtitle}[1]{
  \posttitle{
    \begin{center}\large#1\end{center}
    }
}

\setlength{\droptitle}{-2em}

  \title{2018-11-15 qbic test 2a}
    \pretitle{\vspace{\droptitle}\centering\huge}
  \posttitle{\par}
    \author{Theresa Swayne}
    \preauthor{\centering\large\emph}
  \postauthor{\par}
      \predate{\centering\large\emph}
  \postdate{\par}
    \date{11/16/2018}


\begin{document}
\maketitle

\subsection{Testing nuclear detection and DNA content
analysis}\label{testing-nuclear-detection-and-dna-content-analysis}

The background-corrected integrated intensity (``total intensity'') of
nuclei should be proportional to DNA content.

The DNA content in a mixed population of actively cycling cells should
form a roughly bimodal distribution representing 1N and 2N populations,
with some intermediate cells in S phase.

CellProfiler pipelines were used to perform background correction,
identify nuclei using DAPI, and measure intensity of DAPI and other
channels in a set of 100 images collected on 11-5-2018.

\begin{verbatim}
## Parsed with column specification:
## cols(
##   .default = col_double(),
##   ImageNumber = col_integer(),
##   ObjectNumber = col_integer(),
##   Metadata_FileLocation = col_character(),
##   Metadata_Frame = col_integer(),
##   Metadata_Series = col_integer(),
##   Metadata_XY = col_integer(),
##   Metadata_basename = col_character(),
##   AreaShape_Area = col_integer()
## )
\end{verbatim}

\begin{verbatim}
## See spec(...) for full column specifications.
\end{verbatim}

\begin{verbatim}
## 922 cells detected.
\end{verbatim}

Plotting a histogram of the DNA content over all ----- nuclei detected:

\begin{Shaded}
\begin{Highlighting}[]
\CommentTok{# use echo=FALSE to prevent printing code}
\KeywordTok{hist}\NormalTok{(dnaTotalInt, }\DataTypeTok{breaks =}\NormalTok{ (}\DecValTok{0}\OperatorTok{:}\DecValTok{50}\NormalTok{))}
\end{Highlighting}
\end{Shaded}

\includegraphics{2018-11-15_qbic_test2a_files/figure-latex/DNA content-1.pdf}

This shows that there is a roughly bimodal distribution with peaks
\textasciitilde{} 12 and \textasciitilde{} 24, with tails on either
side.

\begin{Shaded}
\begin{Highlighting}[]
\NormalTok{low_vals <-}\StringTok{ }\KeywordTok{mean}\NormalTok{(dnaTotalInt }\OperatorTok{<}\StringTok{ }\DecValTok{1}\NormalTok{) }\OperatorTok{*}\StringTok{ }\DecValTok{100}
\NormalTok{high_vals <-}\StringTok{ }\KeywordTok{mean}\NormalTok{(dnaTotalInt }\OperatorTok{>}\StringTok{ }\DecValTok{30}\NormalTok{) }\OperatorTok{*}\StringTok{ }\DecValTok{100}
\NormalTok{central_vals <-}\StringTok{ }\KeywordTok{mean}\NormalTok{(dnaTotalInt }\OperatorTok{>}\StringTok{ }\DecValTok{1} \OperatorTok{&}\StringTok{ }\NormalTok{dnaTotalInt }\OperatorTok{<}\StringTok{ }\DecValTok{30}\NormalTok{) }\OperatorTok{*}\StringTok{ }\DecValTok{100}

\KeywordTok{cat}\NormalTok{(central_vals, }\StringTok{"% of the values are between 1 and 30; "}\NormalTok{,low_vals,}\StringTok{"% <1; and"}\NormalTok{,high_vals,}\StringTok{"% >30"}\NormalTok{)}
\end{Highlighting}
\end{Shaded}

\begin{verbatim}
## 97.2885 % of the values are between 1 and 30;  0 % <1; and 2.711497 % >30
\end{verbatim}

Ultimately we would like to see an inverted horseshoe-type plot of EdU
(showing S phase cells) vs.~DNA content where the cells incorporating
EdU are primarily those between the 2 peaks.

Generating the other dimension of the plot from the corrected mean
intensity from Channel 2, EdU:

\begin{Shaded}
\begin{Highlighting}[]
\CommentTok{#EdUMeanInt <- qbic_C1DNANuclei$Intensity_MeanIntensity_Channel2EdU}
\NormalTok{EdUMeanInt <-}\StringTok{ }\NormalTok{qbic_C1DNANuclei}\OperatorTok{$}\NormalTok{Intensity_MeanIntensity_Channel2EdUCorr}
\KeywordTok{hist}\NormalTok{(EdUMeanInt)}
\end{Highlighting}
\end{Shaded}

\includegraphics{2018-11-15_qbic_test2a_files/figure-latex/reading EdU data-1.pdf}

\begin{Shaded}
\begin{Highlighting}[]
\KeywordTok{plot}\NormalTok{(}\DataTypeTok{x =}\NormalTok{ dnaTotalInt, }\DataTypeTok{y =}\NormalTok{ EdUMeanInt) }\CommentTok{# basic scatter plot}
\end{Highlighting}
\end{Shaded}

\includegraphics{2018-11-15_qbic_test2a_files/figure-latex/reading EdU data-2.pdf}

\subsubsection{Are nuclei detected
properly?}\label{are-nuclei-detected-properly}

Several images were selected at random, and the image of detected
objects was compared with the original data.

False positives (non-cells counted as objects) and false negatives
(cells not included in the object map) were scored manually.

\begin{figure}
\centering
\includegraphics{/Users/confocal/Desktop/output/test2-3/detection_Montage.png}
\caption{Detection of nuclei in XY point 30, 11/5/18. Left, original
DAPI channel. Right, detected objects, each in a different color. Red
circle shows a false positive.}
\end{figure}

\begin{Shaded}
\begin{Highlighting}[]
\NormalTok{qbic_check <-}\StringTok{ }\KeywordTok{read_csv}\NormalTok{(}\StringTok{"~/github_theresaswayne/cellprofiler/2018-11-15 qbic test 2a.csv"}\NormalTok{)}
\end{Highlighting}
\end{Shaded}

\begin{verbatim}
## Parsed with column specification:
## cols(
##   Image = col_integer(),
##   `Cells detected by CP` = col_integer(),
##   `False positives` = col_integer(),
##   `False negatives` = col_integer(),
##   Notes = col_character()
## )
\end{verbatim}

\begin{Shaded}
\begin{Highlighting}[]
\KeywordTok{kable}\NormalTok{(qbic_check[,(}\DecValTok{1}\OperatorTok{:}\DecValTok{5}\NormalTok{)])}
\end{Highlighting}
\end{Shaded}

\begin{longtable}[]{@{}rrrrl@{}}
\toprule
Image & Cells detected by CP & False positives & False negatives &
Notes\tabularnewline
\midrule
\endhead
89 & 9 & 0 & 0 & NA\tabularnewline
51 & 11 & 0 & 0 & NA\tabularnewline
31 & 12 & 0 & 0 & NA\tabularnewline
36 & 9 & 2 & 0 & NA\tabularnewline
45 & 0 & 0 & 0 & NA\tabularnewline
13 & 11 & 0 & 0 & NA\tabularnewline
2 & 7 & 0 & 0 & CP count based on looking at obj image for this point
and below\tabularnewline
12 & 9 & 0 & 0 & NA\tabularnewline
7 & 13 & 0 & 0 & NA\tabularnewline
45 & 0 & 0 & 0 & NA\tabularnewline
43 & 16 & 0 & 0 & 1 possible false pos (out of focus) lower
right\tabularnewline
31 & 12 & 0 & 0 & Edge objects inconsistent -- here a cell was detected
that is at edge (partial nucleus) but in \#43 a cell barely on edge was
eliminated\tabularnewline
32 & 15 & 0 & 0 & NA\tabularnewline
29 & 10 & 0 & 0 & NA\tabularnewline
30 & 14 & 1 & 0 & false pos = elongated junk\tabularnewline
39 & 9 & 0 & 0 & NA\tabularnewline
68 & 15 & 0 & 0 & NA\tabularnewline
85 & 8 & 0 & 0 & NA\tabularnewline
65 & 3 & 1 & 0 & NA\tabularnewline
86 & 8 & 0 & 0 & one cell near top edge was segmented smaller than in
reality\tabularnewline
\bottomrule
\end{longtable}

\begin{Shaded}
\begin{Highlighting}[]
\NormalTok{total_cells <-}\StringTok{ }\KeywordTok{sum}\NormalTok{(qbic_check}\OperatorTok{$}\StringTok{`}\DataTypeTok{Cells detected by CP}\StringTok{`}\NormalTok{)}
\NormalTok{false_pos <-}\StringTok{ }\DecValTok{100} \OperatorTok{*}\StringTok{ }\KeywordTok{sum}\NormalTok{(qbic_check}\OperatorTok{$}\StringTok{`}\DataTypeTok{False positives}\StringTok{`}\NormalTok{)}\OperatorTok{/}\NormalTok{total_cells}
\NormalTok{false_neg <-}\StringTok{ }\DecValTok{100} \OperatorTok{*}\StringTok{ }\KeywordTok{sum}\NormalTok{(qbic_check}\OperatorTok{$}\StringTok{`}\DataTypeTok{False negatives}\StringTok{`}\NormalTok{)}\OperatorTok{/}\NormalTok{total_cells}
\KeywordTok{cat}\NormalTok{(}\StringTok{"Out of"}\NormalTok{,total_cells,}\StringTok{"cells detected, there were"}\NormalTok{,false_pos,}\StringTok{"% false positives and"}\NormalTok{,false_neg,}\StringTok{"% false negatives."}\NormalTok{)}
\end{Highlighting}
\end{Shaded}

Out of 191 cells detected, there were 2.094241 \% false positives and 0
\% false negatives.

\subsection{Question: What to do with apparent
clumps?}\label{question-what-to-do-with-apparent-clumps}

Some DAPI structures appear to have multiple lobes. Are these
multi-lobed nuclei, mitotic cells, or actually multiple cells close
together? If needed, we can modify the object detection to split objects
by shape.

\begin{figure}
\centering
\includegraphics{/Users/confocal/Desktop/output/test2-3/clump_Montage.png}
\caption{Examples of clumped nuclei from XY points 21 and 30, 11/5/18.}
\end{figure}

\subsection{Summary}\label{summary}

\begin{itemize}
\tightlist
\item
  Time of processing with CellProfiler is \textless{} 10 min per 100
  images for illumination correction and the same for object detection.
\item
  Nuclei are detected well in this system. There are virtually no false
  negatives.
\item
  Some bright debris results in false positives, about 2\%.
\item
  We need to decide if some objects should be split or if they are
  actually single nuclei.
\end{itemize}


\end{document}
